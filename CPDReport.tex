% Please do not change the document class
\documentclass{scrartcl}

% Please do not change these packages
\usepackage[hidelinks]{hyperref}
\usepackage[none]{hyphenat}
\usepackage{setspace}
\doublespace

% You may add additional packages here
\usepackage{amsmath}

% Please include a clear, concise, and descriptive title
\title{Personal Reflection}

% Please do not change the subtitle
\subtitle{COMP230 - CPD Report}

% Please put your student number in the author field
\author{1706966}

\begin{document}

\maketitle

\section{Introduction}
This report will find key skills that I need to look at improving, with the aim of helping me reach my aspirations for a future career. My current aspirations are a bit up in the air, working through this course and learning more about the game industry is admittedly putting me off working in it. 

I still have an interest in working within the industry but leaning towards joining smaller companies hopefully for an extended period. Alternatively, I am considering programming outside of the game specific area where most skills learnt will still be relevant. My final path that I might walk is moving back into working in the IT sector, using my new skills to supplement that work, and creating games as a side project. 

\section{Receptiveness to Feedback - Interpersonal}
I would like to look deeper at my receptiveness to feedback. Specifically how I respond to negative. I can think of one instance where this happened as it has stuck with me. Not considering this continuously may allow for similar experiences to occur. When I receive negative feedback I have an insatiable urge to quickly explain my side. Although this comes with the best of intentions, wanting to make the other person see the reasons for my choices, I believe it likely comes off as being very defensive and argumentative. 

All my aspirations require me to work with others and receive feedback over the years, therefore it is important to work on this skill to ensuring I can learn well from feedback.
\subsection{Action}
If I find myself in a situation where I do not fully understand a piece of feedback given, I will question it without using defensive language. I will fill out a spreadsheet specifying where I succeed and fail at this. Having this goal in mind will provide more thoughtful interactions, meaning that I don't come off as defensive but still interested in useful feedback. I will do this until the end of this academic year.

\section{Programming Patterns - Cognitive}
Although programming patterns are something I need to look more closely at, it has been a failure on my part not looking into these as much as I feel is necessary. I believe having knowledge of these patterns will give me effective and efficient solutions to implement often throughout my future code base. I expect that not having a decent grasp of these patterns will look bad for me when searching for a job in the game industry.
\subsection{Action}
I will read one section of gameprogrammingpatterns.com every week. To measure this I will keep a list of completed sections and the week they were read on. With my current knowledge of programming, I should be able to understand the patterns well. Doing this will give me useful solutions to future problems, and help me come across as more employable for software related jobs. Reading a section each week will mean I complete this within twenty weeks.

\section{Debugging - Procedural}
A key area in which I would like to improve on is what the different sections of an IDE can do for me, specifically in regards to debugging. I have only learnt how to use these types of software by seeing it used, which has given me a basic working knowledge but there is a lot more that can be achieved were I to have a deeper understanding of the tools available. Although in the future I will most certainly be using different IDE's, the process of learning about them and what they can do will likely be similar. I believe it will be good practice for myself to start now and research each IDE that I use.
\subsection{Action}
I will read through the documentation for Visual Studio to find its inbuilt tools useful for debugging. I will make a note of each new possible tool with an aim to find at least six. I have a base understanding of Visual Studio but finding out about the abilities it has will likely provide me with smarter ways to debug issues. This will continue to be useful throughout my career as many IDE's have similar abilities. I will have looked through the documentation by the end of this study block.

\section{Considering Team-Mates Morale - Affective}
This section looks at how I have seen positive feedback to play a large part in invigorating my peers to create fantastic work. Often after receiving well deserved positive feedback for I have seen my team work with a renewed vigour.

I have made attempts to provide my peers with this, but I have seen myself dwindle down from the amount it happens. A great opportunity to do this which I have yet to capitalise on is in the PO meetings with the anonymous feedback. Making my peers feel accomplished with their work may not help my work specifically, but when working in a team environment doing something for the greater good of the team should have worth in itself. Being able to do this well will hopefully provide better end products throughout the terms and in a team job environment.
\subsection{Action}
I will give everyone some positive feedback where applicable during anonymous PO reviews. I will then make a note of every PO review meeting where I manage to supply this positive feedback, and to how many members of the team it was provided. I have a good team that works hard so finding good things to say should be achievable. This will hopefully drive the team to continue to supply good work and improve overall morale. I will continue this goal until the end of this academic year.

\section{Project Tracking - Dispositional}
Balancing my work throughout the year has presented the new challenge of running the risk that I become overly focused and forget how projects relate to assignment briefs. I'm very aware of how poor my memory can be, so I should implement a way to combat this flaw. Making sure I always have an idea of what all my assignments require will allow me to correctly allocate time to each section across the term, hopefully meaning that I work smarter and no sections get missed until close to their deadlines. This ability to remind myself and gain a constant knowledge of the jobs that need to be completed, will hopefully flow fantastically into a working environment and even improve my ability to organise my personal life.
\subsection{Action}
I will create a Trello board to break down assignments into months and check it weekly to refresh my memory. I will have a tracker on the Trello that I mark off each week to track my progress. I have used Trello and like how it can display tasks, so I will be comfortable with this method. This will help me build new working practices to combat my failing memory, these practices can be taken into any working environment. This will be an ongoing goal until the end of this academic year.

\section{Conclusion}
Everything selected not only has its part to play in improving my university work but will also set me up to achieve well in any of the roles I aspire to reach. Having recognised how I have achieved previously outlined CPD goals this year, I look forward to achieving these ones, and seeing what I may discover about myself in this coming term. 

\end{document}
