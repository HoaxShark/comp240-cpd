% Please do not change the document class
\documentclass{scrartcl}

% Please do not change these packages
\usepackage[hidelinks]{hyperref}
\usepackage[none]{hyphenat}
\usepackage{setspace}
\doublespace

% You may add additional packages here
\usepackage{amsmath}

% Please include a clear, concise, and descriptive title
\title{Personal Reflection}

% Please do not change the subtitle
\subtitle{COMP240 - CPD Report}

% Please put your student number in the author field
\author{1706966}

\begin{document}

\maketitle

\section{Introduction}
This report will identify key skills that I need to look at improving, with the aim of helping me reach my aspirations for a future career.

\section{Providing Critical Feedback - Interpersonal}
It can be a difficult thing to provide critical feedback to my peers, I do this well when it comes to other programmers but with regards to people outside of my speciality, it feels a bit like intruding when I don't have good knowledge of the subject. 

It's been brought up by some successful third-year teams that having someone willing to give completely open criticism even when they are not an expert on the subject has been incredibly beneficial to their projects. I have experienced this during times when I felt art assets could be changed to fit more with the project's style, but felt too out of place to give this feedback, only for it to be given by our PO later. 

Being someone that is comfortable giving useful and succinct feedback in a friendly manner will be great for not only me but also helps others find ways to improve. I feel like this type of person can be invaluable to a team and end up being seen in high regard assuming they can supply that feedback in a polite manner.
\subsection{Action}
During each sprint meeting next term where we review the jobs that have been completed, I will make sure to always give my honest opinion about the work with an awareness that I might not necessarily have an expert opinion. I will keep a spreadsheet of this for each week where I can mark down if I failed or succeeded. Getting used to giving this feedback will hopefully help other team members spot possible changes that they might have missed and make it easier for me to do in a professional environment.

\section{C++ Memory Management - Cognitive}
I haven't spent as much time as I would like practising skills in C++. The need to more actively manage memory manually in this language compared to others has been a stumbling point on one of my projects. When creating objects and not deleting them correctly in a project this year, I ended up losing marks due to my lack of knowledge and the application didn't work as optimally as it could have. I knew this was a problem but found myself unable to find the solution.
These flaws wouldn't be acceptable in a professional environment where every bit of memory for a project is very important, not to mention the problems it could cause for users if handled incorrectly. 
\subsection{Action}
I have found two useful resources that teach about memory management \cite{c++programming/memorymanagement} \cite{whateveryprogrammershouldknowaboutmemory}. Each week I will spend 2 hours reading through these sites until I have covered all the topics listed. I will make a list of all the topics and check them off as I complete them. Going through these resources should give me a better understanding of how computer memory works and how to successfully manage it in C++, therefore helping me to write well-optimised code in the future.

\section{UML Diagrams - Procedural}
I have grown a new respect for UML diagrams over the past year, I still can find them difficult to create as it can be tricky for me to plan ahead how things will interact. My knowledge of the subject and experience in making them is still growing and therefore the process is time-consuming. However, I see the value in having a base plan before delving into coding and would really like to improve my ability to create these diagrams efficiently. Another key reason to improve this skill is that being able to show my processes in UML format will be great when applying for jobs and trying to show off my ideas to future work colleagues.
\subsection{Action}
For each new algorithm I design during the next term I will create relevant UML diagrams to be shown on the readme.md of the project. These can then be used to measure my progress by checking that the diagrams exist for my projects. This will provide me with good experience in creating these diagrams and make it easier and faster to do in the future.

\section{How Do I Feel? - Affective}
When it comes to considering the affective domain I have always struggled the most. It's very uncommon for me to stop and think about emotions or mental states, so when I do it doesn't come very naturally. Having a better ability to comprehend how different states can be affecting myself and other people will make communicating in work environments easier. It should also give me better mental well-being and understanding of myself, allowing me to enjoy my work more and avoid situations that put me in a negative state.
\subsection{Action}
At the end of each week during the next term, I will set aside 30 minutes of time to write a weekly reflection, looking specifically for any outlying differences to my normal emotional state. This will give me a better understanding of my own emotions and the reasons behind why they have been affected, as well as helping me get used to reflective writing, improving my future CPD's. I will add this to my weekly to-do list and check off each session as I complete them.

\section{Job List Time Allocation - Dispositional}
My ability to organise my time and manage my projects over a large time scale has improved greatly by following previous smart goals I have created. However, I still see room for improvement. The new area of fault lies in not splitting my work for the day down well enough, I have often had multiple jobs to achieve in a day and end up getting lost in completing only one of these jobs to a high standard. Although this overall gives me a high quality of work it has sometimes left other team members waiting on a job to be completed. This gives an overall slowdown in work for the project which is key to avoid, as it can not only annoy team members but also halt progress in certain areas.
\subsection{Action}
At the start of each day I will separate out my jobs as normal but now allocate time to each job. This will help me move on to the next job and not stay working on the previous one, making sure that all jobs are getting an adequate amount of work done on them. I will keep a record of these to-do lists and mark them as successful or over-ran depending on how the day went.

\section{Conclusion}
Everything selected not only has its part to play in improving my university work but will also set me up to achieve well in any of the future careers I aspire to work in. Having recognised how I have achieved previously outlined CPD goals this year, I look forward to achieving these ones and seeing what I may discover about myself in the coming term. 

	\bibliographystyle{ieeetran}
	\bibliography{references}

\end{document}
